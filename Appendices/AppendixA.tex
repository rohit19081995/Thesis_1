% Appendix A

\chapter{Calculation of Required Junction Resistance} % Main appendix title

\label{AppendixA} % For referencing this appendix elsewhere, use \ref{AppendixA}

If the capacitive pads of the transmon have a $70$fF shunt capacitance ($C_s$)\footnote{The shunt capacitance $C_s$ is also the dominating capacitance for a junction with small area ($C_s\gg C_j$)} as in \cite{Paik2011}, then the charging energy is
\begin{equation}
\frac{E_C}{h} = \frac{e^2}{2C_\Sigma}= \frac{e^2}{2*\SI{70}{\femto \farad}} = \SI{0.277056}{\giga\hertz}
\end{equation}

To construct a $\SI{7}{\giga\hertz}$ qubit, $E_2-E_1 = \sqrt{8E_JE_C}-E_C = h\SI{7}{\giga\hertz}$. Then the Junctiion energy is
\begin{equation}
\frac{E_J}{h} = \frac{(h\times\SI{7}{\giga\hertz}+E_C)^2}{8E_C} = \SI{23.89}{\giga\hertz}
\end{equation}

The critical current for a given $E_J$ can be calculated using \ref{eqn:junction energy}.
\begin{equation}
I_0 = \frac{2eE_J}{\hbar} = \SI{46.2442}{\nano\ampere}
\end{equation}

The Juntion Resistance ($R_N$) can be calculated using the Ambegaokar-Baratoff formula
\begin{equation}
I_0 = \frac{\pi\Delta_s}{2eR_N}\tanh\frac{\Delta_s}{2k_BT}
\end{equation}
where $k_B$ is the Boltzmann Constant and $\Delta_s$ is the superconducting gap, which in this case is that of aluminium. This can be calculated using BCS theory as
\begin{equation}
\Delta_s = 1.764\times k_BT_C
\end{equation}
where $T_C = \SI{1.1}{K}$ si the critical temperature of aluminium.

In the zero temperature limit, the junction resistance $R_N$ is given by
\begin{equation}
R_N = \frac{\pi\Delta_s}{2eI_0} = \SI{5.68188}{\kilo\ohm}
\end{equation}

The junction inductance can also be calculated as
\begin{equation}
L_J = \frac{\Phi_0}{I_0} = \frac{\hbar}{2eI_0} = \SI{7.09833}{\nano\henry}
\end{equation}
