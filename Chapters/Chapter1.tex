% Chapter 6

\chapter{Introduction} % Main chapter title

\label{Chapter1} % For referencing the chapter elsewhere, use \ref{Chapter1} 

The key to improving computing performance for the past few years has been to reduce the size of the transistors used in the process, but this process cannot continue for much longer since the effects of quantum mechanics will prevent the further reduction of transistor size.

In 1982, Richard Feynman suggested that we could use the effects of quantum mechanics to our advantage and build a "quantum computer". For a few decades, quantum computers were only of theoretical interest. Many quantum algorithms were developed which could perform  certain tasks exponentially faster than their classical counterparts.

In 1998, the first Nuclear Magnetic Resonance (NMR) quantum computers were implemented.

Cavity QED, which is the study of light in a superconducting reflective cavity interacting with atoms, can be used for many physical applications including quantum computing and has been studied since the 1940s.

Circuit QED, which is basically cavity QED where superconducting circuits are used instead of a reflective cavity. Microwave photons are used instead of optical light and {\JJ}s are used instead of atoms. The {\JJ}s have a transition frequency which is of the same order as microwaves.

This thesis describes the use of a superconducting cavity as a microwave resonator, and discusses it's interaction with  an implementation of a qubit, called a transmon.